\documentclass[a4paper,12pt]{article}
\usepackage[latin1]{inputenc}
\usepackage{listings}
\usepackage{color}
\usepackage{a4wide}
\usepackage{graphicx}
\usepackage{subfig}
\usepackage{wrapfig}
\usepackage{hyperref}
\usepackage{amsmath}
\usepackage{tikz}
\usetikzlibrary{shapes, arrows}
\usepackage{appendix}
\usepackage{amsmath}
\usepackage{tablefootnote}
\usepackage{tabto}

\usepackage{bm, amssymb, pifont, pgfplots, multirow, float, soul, color, verbatim, indentfirst}

\newcommand{\be}{\begin{equation}}
\newcommand{\ee}{\end{equation}}
\lstset{
   breaklines=true,
   basicstyle=\ttfamily
}
\usepackage[figurewithin=section,tablewithin=section]{caption}
\setcounter{secnumdepth}{4}
\pagestyle{headings}
\begin{document}
\title{\textbf{Cavity Model}}
\author{Jorge Alberto Diaz Cruz}
\date{\today}
\numberwithin{equation}{section}
\maketitle
\setcounter{tocdepth}{2}
\tableofcontents
\newpage
\section{From math to code}
This section describes the mathematical manipulation of the cavity's differential equation so it can be implement in python code. Let's start with a set of two first order diferential equations derived in \cite{ref:LCLS-II System Simulations}:

\begin{equation}
  \frac{d\vec{S}_{\mu}}{dt} = -\omega_{f_{\mu}}\vec{S}_{\mu} + \omega_{f_{\mu}}e^{-j\theta_{\mu}}\left(2\vec{K}_g\sqrt{R_{g_{\mu}}}
    - R_{b_{\mu}}\vec{I}_{\rm beam} \right)
\label{eq: dS_dt}
\end{equation}	

\begin{equation}
\frac{d\theta_{\mu}}{dt} = \omega_{d_\mu}
\label{eq: d0_dt}
\end{equation}

Where the subcript $\mu$ represents each eigenmode of the cavity and:

\begin{equation}
\vec{V}_{\mu} = \vec{S}_{\mu}e^{j\theta_{\mu}}
\label{eq:V_mu}
\end{equation}
\\
$\omega_{f_{\mu}}$:\tab Cavity bandwidth
\\$\vec K_{\rm g}$:\tab Incident wave amplitude in $\sqrt{\rm Watts}$
\\$R_{\rm g_{\mu}}=Q_{\rm g_{\mu}}(R/Q)_{\mu}$:\tab Coupling impedance of the drive port
\\$\vec I_{\rm beam}$:\tab Beam current
\\$R_{\rm b_{\mu}}=Q_{\rm L_{\mu}}(R/Q)_{\mu}$:\tab Coupling impedance to the beam
\\$\omega_{d_{\mu}}=2\pi\Delta f_{\mu}$:\tab Detune frequency
\\
\\Let's now define:
\\
\\$a=2\sqrt{R_{g_{\mu}}}\omega_{f_{\mu}}$
\\$b=R_{b_{\mu}}\omega_{f_{\mu}}$
\\$c=\omega_{f_{\mu}}$
\\
\\So we can write equation ~\ref{eq: dS_dt} as:

\begin{equation}
  \frac{d\vec{S}_{\mu}}{dt} = -c\vec{S}_{\mu} + e^{-j\theta_{\mu}}\left(a\vec{K}_g - b\vec{I}_{\rm beam} \right)
\label{eq: dS_dt_reduced}
\end{equation}

For a complex $\vec{S}_{\mu}=S_r+jS_i$ we can split equation \ref{eq: dS_dt_reduced} in the real and imaginary parts:

\begin{equation}
  \frac{d\vec{S}_{\mu}}{dt} = -c(S_r+jS_i) + (cos\theta-jsin\theta)(a\vec{K}_g - b\vec{I}_{\rm beam})
\label{eq: dS_dt_reduced_Scomplex}
\end{equation}

\begin{equation}
  \frac{dS_r}{dt} = (a\vec{K}_g - b\vec{I}_{\rm beam})cos\theta-cS_r
\label{eq: dSr_dt_reduced_Scomplex}
\end{equation}

\begin{equation}
  \frac{dS_i}{dt} = -(a\vec{K}_g - b\vec{I}_{\rm beam})sin\theta-cS_i
\label{eq: dSi_dt_reduced_Scomplex}
\end{equation}

If we also assume a complex incident wave $\vec{K}_g=K_r+jK_i$ we have:

\begin{equation}
  \frac{dS_r}{dt} = (aK_r - b\vec{I}_{\rm beam})cos\theta-cS_r+aK_isin\theta
\label{eq: dSr_dt_reduced_S_K_complex}
\end{equation}

\begin{equation}
  \frac{dS_i}{dt} = -(aK_r - b\vec{I}_{\rm beam})sin\theta-cS_i+aK_icos\theta
\label{eq: dSi_dt_reduced_S_K_complex}
\end{equation}

Equations \ref{eq: d0_dt}, \ref{eq: dSr_dt_reduced_S_K_complex} and \ref{eq: dSi_dt_reduced_S_K_complex} can be implemendted in python using the following code:

{\fontfamily{qcr}\selectfont\small
\# Define Cavity model
\\def cavity(z, t, RoverQ, Qg, Q0, Qprobe, bw, Kg\_r, Kg\_i, Ib,
          foffset):
\\
\\        Rg = RoverQ * Qg
\\        Kdrive = 2 * np.sqrt(Rg)
\\        Ql = 1.0 / (1.0/Qg + 1.0/Q0 + 1.0/Qprobe)
\\        K\_beam = RoverQ * Ql
\\        w\_d = 2 * np.pi * foffset
\\
\\        a = Kdrive * bw
\\        b = K\_beam * bw
\\        c = bw
\\
\\        yr, yi, theta = z
\\
\\        dthetadt = w\_d
\\        dydt\_r = (a * Kg\_r - b * Ib)*np.cos(theta) - (c * yr) + a * Kg\_i * np.sin(theta)
\\        dydt\_i = -(a * Kg\_r - b * Ib)*np.sin(theta) - (c * yi) + a * Kg\_i * np.cos(theta)
\\        return dydt\_r, dydt\_i, dthetadt
}

\newpage
\begin{thebibliography}{19}   % Use for  1-9  references
\bibitem{ref:LCLS-II System Simulations}
LBNL LLRF team, ``LCLS-II System Simulations: Physics,''
October 7, 2015.
\end{thebibliography}

\end{document}
